\documentclass[11pt]{article}

\usepackage{amsmath}
\usepackage{amsfonts}
\usepackage{stmaryrd}

\newcommand{\klet}[3]{\textsf{let}~#1=#2~\textsf{in}~#3}
\new	command{\kfst}[1]{\textsf{fst}(#1)}
\newcommand{\ksnd}[1]{\textsf{snd}(#1)}
\newcommand{\kproject}[1]{\textsf{project}(#1)}
\newcommand{\kconstruct}[1]{\textsf{construct}(#1)}
\newcommand{\kinvert}[1]{\textsf{invert}(#1)}
\newcommand{\ebool}{\llbracket \mathbb{B} \rrbracket}
\newcommand{\eint}[1]{\llbracket \mathbb{Z}_{2^{#1}} \rrbracket}
\newcommand{\tunit}{\textsf{unit}}
\newcommand{\Tunit}{\textsf{Unit}}

\begin{document}

\begin{figure}
\begin{align*}
  t :=&~t \land t ~|~ t \oplus t ~|~ (t, t) ~|~ \kfst{t} ~|~ \ksnd{t} ~|~ \klet{x}{t}{t} ~|~ x ~|~ \tunit\\
  T :=&~\ebool~|~T \times T~|~\Tunit\\
  \land, \oplus \in&~(\ebool\times\ebool) \rightarrow \ebool\\
\end{align*}
\caption{%
  The base language of circuits.
}\label{fig:base-lang}
\end{figure}

\begin{figure}
\begin{align*}
  &add_2(a, b) :=\\ 
  &~~\klet{((a_0, a_1), (b_0, b_1))}{(a, b)}{\\}
  &~~\klet{out_0}{a_0 \oplus b_0}{\\}
  &~~\klet{carry}{a_0 \land b_0}{\\}
  &~~\klet{out_1}{a_1 \oplus (b_1 \oplus carry)}{\\}
  &~~(out_0, out_1)
\end{align*}

\caption{%
  An example circuit, a two bit adder, written in the base language.
}
\end{figure}

\begin{figure}
\begin{align*}
  t :=&~\ldots~|~\kproject{t}~|~\kconstruct{t}~|~t+t~|~t\cdot t~|~\kinvert{t}\\
  T :=&~\ldots~|~\eint{n}\\
  T^0 :=&~ \unit\\
  T^n :=&~ T \times T^{n-1}\\
  \textsf{project} \in&~\eint{n} \rightarrow \ebool^n\\
  \textsf{construct} \in&~\ebool^n\rightarrow\eint{n}\\
  +, \cdot\in&~(\eint{n} \times \eint{n}) \rightarrow \eint{n}\\
  \textsf{invert} \in&~\eint{n} \rightarrow \eint{n}\\
\end{align*}
\caption{%
  The base language extended with ring operations (addition, multiplication, and additive inverse) as well as conversions between ring representation and Boolean representation (project and construct).
}\label{fig:base-lang}
\end{figure}



\end{document}
